\documentclass{article}
\usepackage[utf8]{inputenc}
\usepackage[pdftex]{hyperref}
\usepackage{hyperref}
\usepackage{graphicx}
\graphicspath{ {./images/} }
\usepackage{amsmath}
\usepackage{indentfirst}

\title{Mexico}
\author{Yolanda Nogueira e Ana Luiza Nunes}
\date{September 2020}

\begin{document}
\maketitle
\section{Introduction}

Buscamos em diversos artigos e notícias as relações entre o processo de espalhamento do covid-19 e as características externas diretas e indiretas que pudessem alterar o nível de contágio no país. 
Um dos artigos que encontramos (\href{https://www.sciencedirect.com/science/article/pii/S0048969720330771}{Artigo 1}) descreve a importância de olharmos para o clima da região e de relacionarmos com o tema. 

Esse artigo mostra a associação entre a quantidade de casos confirmados do vírus e as características climáticas (em conjunto com os anúncios diários de meteorologia) do México na fase 1 da pandemia. Essa conexão é estatisticamente significante. Fatores meteorológicos influenciaram a tendência de surtos regionais no país. Acreditamos ser pela suscetibilidade do hospedeiro durante o inverno frio. Então, as características climáticas desempenharam um papel crucial na infecção local durante a fase 1 sendo as regiões temperadas (predominante) mais vulneráveis que as secas ou tropicais.

\begin{center}
    "The tropical climate (mean temperature
around 25.95 $^o$C and mean precipitation around 8.74 mm) delayed the regional onset. However, the regional
onset in dry climates emerged earlier as consequence of the lower temperatures and higher precipitations
(20.57 $^o$C and 20.87 mm respectively) than the observed in the tropical climate. The fastest regional onsets
were observed in tempered climates in states where the lowest temperatures and lowest precipitations were
registered (19.65 $^o$C and 8.48 mm respectively)."   - \href{https://www.sciencedirect.com/science/article/pii/S0048969720330771}{(Artigo 1)}
\end{center}

Outro tópico que achamos interessante abordar nessa introdução é o fato de que o México é o país com a menor taxa de testes entre os países da OCDE (Organização para a Cooperação e Desenvolvimento Economico - 37 países membros). 

\begin{center}
    "O governo rechaçou o programa de testagem em massa, sob o pretexto de ser caro e inútil, de acordo com o subsecretário de Saúde, Hugo López- Gattel, que lidera a força-tarefa do coronavírus. Mas o prognóstico é pior: sem dados, não há controle sobre a doença. Um levantamento de certidões de óbito realizado pela ONG Mexicanos Contra a Corrupção e a Impunidade entre 18 de março e 12 de maio constatou que o número de mortes por suspeitas de Covid-19 na Cidade do México é o triplo do que divulgado pelo governo." - \href{https://g1.globo.com/mundo/blog/sandra-cohen/post/2020/06/05/mexico-sai-da-quarentena-no-pico-da-pandemia.ghtml}{G1}
\end{center}

A testagem é recomendada para o desenvolvimento de um planejamento mais preciso de quando suspender as medidas de intervenção (quarentena e outras) em vigor. Além disso, o teste é necessário para a estimação do verdadeiro tamanho da epidemia. 

Tais medidas intervencionistas foram implementadas na região. Medidas opcionais (apenas recomendadas) de distanciamento social para o setor privado foram publicadas no dia 23 de março de 2020 e para o setor público (suspensão de atividades não essenciais) dia 26 de março. Inclusive, várias matérias falam sobre o afrouxamento das medidas de prevenção contra a pandemia no país. 

Através de dados recolhidos pela google, encontramos informações sobre os dados de mobilidade que mostra a diminuição drástica de ações fora das residências dos mexicanos após a medida opcional ser anunciada. Essa pesquisa se encontra nos gráficos abaixo (as legendas indicam a área do estudo).

\includegraphics[width=10.5cm, height=5cm]{20200926_220642.jpg}

\includegraphics[width=10.5cm, height=5cm]{20200926_220710.jpg}

\includegraphics[width=10.5cm, height=5cm]{20200926_220758.jpg}

\includegraphics[width=10.5cm, height=5cm]{20200926_220834.jpg}

(Pensamos em incluir esses dados futuramente na nossa modelagem. Mas, sendo uma abordagem muito complexa, vamos pensar muito se será realmente viável. Falaremos disso em outra seção.)

Voltando para o problema da escassez de testes, o \href{https://www.sciencedirect.com/science/article/pii/S0196064420306016}{Artigo 2} fala sobre esse tema e mostra como os serviços de emergência médica podem ajudar na análise de dados. O artigo estuda apenas a cidade de Tijuana, mas a conclusão se encaixa em todo o país (mesmo que em diferentes proporções). 

A contabilidade de mortos pelo vírus é feita, normalmente, por eventos de mortalidade intra-hospitalar. Ou seja, as pessoas que não recebram auxílio hospitalar e morreram com covid não são adicionadas ao número total de mortos pela pandemia. Isso acontece em vários outros países. Na Itália, por exemplo, 58\% dos mortos não tiveram seus atestados de óbito como atendimento final no hospital.

\begin{center}
    "An estimated 194.7 (95\%CI: 135.5-253.9) excess out-of-hospital deaths events occurred during
the peak window (April 14th-May 11th), representing an increase of 145\% (70\%-338\%) compared to
expected levels. During the same window, only 5 COVID-19-positive, out-of-hospital deaths were
reported in official statistics." - \href{https://www.sciencedirect.com/science/article/pii/S0196064420306016}{Artigo 2}
\end{center}

Agora, analisando o panorama geral da pandemia no México, vemos que ele não chegou na segunda onda, pois ainda não saiu da primeira. Falarei mais sobre as ações mexicanas que dificultaram (ainda dificultam) a diminuição do número de casos. Segue o gráfico de casos confirmados tirado do \href{https://ourworldindata.org/coronavirus}{Our World in Data}. 

\includegraphics[width=13cm, height=7cm]{grafico.png}

De janeiro a Agosto, quase 30 mil casos de óbito foram registradas acima do esperado pelas autoridades. Mesmo com ocorridos do tipo, o presidete Andrés Manuel López Obrador alegou "No conjunto das nações afetadas pela pandemia, nós não fomos tão atingidos" após o país atingir a marca de 50 mil mortos pelo vírus. Entretanto, em agosto o país conquistou o terceiro lugar em número de óbitos pelo coronavírus, o que não é um pódium muito legal de se subir. Atrás dos Estados Unidos e do Brasil, o México (seu governante) também negligenciou a importãncia de medidas que ajudam a minimizar os impactos da pandemia, como já visto. 

Hoje (27/09/2020) o país tem 726.431 casos confirmados e 76.243 mortes (\href{https://news.google.com/covid19/map?hl=pt-PT&mid=\%2Fm\%2F0b90_r&gl=PT&ceid=PT\%3Apt-150}{fonte: google notícias}).

Como o objetivo desse trabalho é a modelagem da pandemia no país que escolhemos, iremos usar as próximas seções para apresentar nossas ideias e para falar dos artigos que usamos como base (todos eles estarão no repositório do trabalho).

\section{Sobre os Artigos}

Os artigos que usamos para fazer um estudo mais direcionado ao país escolhido e à modelagem são: 

\begin{itemize}
    \item Modeling and prediction of COVID-19 in Mexico applying
mathematical and computational models
    \item Modeling behavioral change and COVID-19 containment in Mexico: A trade-off between lockdown and compliance
    \item Using posterior predictive distributions to analyse
epidemic models: COVID-19 in Mexico City
\end{itemize}
(todos estão no repositório do trabalho).

O primeiro tem uma abordagem computacional (direct and inverse Artificial Neural Network) anexada ao modelo matemático e achamos que esse artigo pudesse ser útil mais tarde para a nossa implementação. Mais para termos uma ideia de como apresentar nossos resultados, já que o método é diferente do que vamos usar.

Os outros dois usam o modelo SEIR modificado e iremos usar esse mesmo esquema na nossa elaboração. Explicaremos melhor na próxima seção. 

\section{Modelagem}
\subsection{Modelo Proposto}
Para modelar os dados da epidemia no México escolhemos o modelo $SEIR$ (Suscetíveis, Expostos, Infecciosos e Recuperados), em que as classes são disjuntas e consideramos a população constante. Entre os contaminados pelo coronavírus, existem aqueles que apresentam severos sintomas e os que não apresentam. Os indivíduos sintomáticos, devido às limitações físicas, se movem menos que os assintomáticos, pensando nisso, consideramos que as taxas de transmissão do vírus são diferentes e separamos a classe Infecciosos em duas novas: $I_a$  $I_s$.

Em dois dos artigos que estudamos foi usada alguma variação do modelo SEIR. No artigo principal 1 foram adicionadas duas classes, hospitalizados ($H$) e indivíduos em estado crítico ($C$), para modelar a demanda de UTI.
No artigo principal 2 foi usada uma versão modificada do modelo SEIR Kermack-McKendrick, em que o modelo dividia a população em dois grupos: o primeiro segue à risca as recomendações de prevenção, incluindo isolamento social, enquanto o segundo não. Nesse mesmo modelo também foi feita a divisão do  compartimento $I$ mencionada anteriormente. Para simplificação do nosso modelo, decidimos não adotar os compartimentos $H$ e $C$ e também não dividimos a população entre os indivíduos isolados e os que não se isolam.

\subsection{Descrição do Modelo}

\includegraphics[width=10.5cm, height=5cm]{FLUXO.png}

\subsubsection{Equações}
\begin{equation}
    S' = -(b_s I_s + b_a I_a )S
\end{equation} \begin{equation}
    E' = (b_s I_s + b_a I_a )S - aE
\end{equation} \begin{equation}
    I_a ' = p a E - c_a I_a
\end{equation} \begin{equation}
    I_s ' = (1-p) a E - c_s I_s
\end{equation} \begin{equation}
    R' = c_a I_a + c_s q I_s
\end{equation} \begin{equation}
    D' = c_s (1 - q) I_s
\end{equation}

\subsubsection{Compartimentos}
\begin{itemize}
    \item $S$: indivíduos suscetíveis à doença;
    \item $E$: indivíduos que foram expostos à doença mas ainda não a transmitem;
    \item $I_a$: indivíduos assintomáticos que transmitem a doença;
    \item $I_s$: indivíduos sintomáticos que transmitem a doença;
    \item $R$: indivíduos que se recuperaram da doença (imunes);
    \item $D$: indivíduos que não sobreviveram à doença.
\end{itemize}
    
\subsubsection{Parâmetros}
\begin{itemize}
    \item $a$: taxa de saída do período latente;
    \item $b_a$: taxa de infecção per capta;
    \item $b_s$: taxa de infecção per capta;
    \item $c_a$: taxa de recuperação;
    \item $c_s$: taxa de saída do estado sintomático;
    \item $p$: probabilidade de ser assintomático;
    \item $q$: probabilidade de recuperação de sintomáticos.
\end{itemize}

\subsubsection{Análise dimensional}
Todos os compartimentos representam pessoas. Utilizaremos $[P]$ na representação. Vamos adotar a unidade de tempo em dias $[T]$.
\begin{itemize}
    \item Possuem unidades $[P]: S, E, I_a, I_s, R, D;$
    \item Possuem unidades $[P][T^{-1}]: S', E', I_a ', I_s ', R', D';$
    \item Possuem unidades $[T^{-1}]: a, c_a, c_s$
    \item Possuem unidades $[P^{-1}][T^{-1}]: b_a, b_s$
\end{itemize}

\subsection{Captura dos dados epidemiológicos}
Pegamos os dados no site Our World In Data - utilizamos o github por ser mais fácil (precisamos apenas fazer um pull no repositório antes de trabalhar com os dados), já que eles fazem uma atualização diariamente.


\section{Bibliografia}
(Links clicáveis)

\begin{enumerate}
    \item \href{https://www.sciencedirect.com/science/article/pii/S0048969720330771}{Artigo 1 - The temperature and regional climate effects on communitarian COVID-19 contagion in Mexico throughout phase 1}
    \item \href{https://www.sciencedirect.com/science/article/pii/S0196064420306016}{Artigo 2 - Excess Out-Of-Hospital Mortality and Declining Oxygen Saturation: The Sentinel 1 Role of EMS Data in the COVID-19 Crisis in Tijuana, Mexico}
    \item \href{https://www.nexojornal.com.br/expresso/2020/03/26/A-abordagem-contradit\%C3\%B3ria-do-M\%C3\%A9xico-para-o-coronav\%C3\%ADrus}{Notícia - A Abordagem Contraditória do México para o Coronavírus}
    \item \href{https://g1.globo.com/mundo/blog/sandra-cohen/post/2020/06/05/mexico-sai-da-quarentena-no-pico-da-pandemia.ghtml}{Notícia - México sai da quarentena no pico da pandemia}
    \item \href{https://noticias.uol.com.br/ultimas-noticias/afp/2020/09/17/cidade-do-mexico-registra-excesso-de-mortalidade-associada-a-covid-19.htm}{Notícia - Cidade do México registra excesso de mortalidade associada à covid-19}
    \item \href{https://noticias.uol.com.br/ultimas-noticias/afp/2020/08/07/lopez-obrador-relativiza-impacto-da-covid-19-no-mexico.htm}{Notícia - Presidente do México relativiza impacto da covid-19; país tem 50 mil mortos}
    \item \href{https://g1.globo.com/podcast/o-assunto/noticia/2020/08/06/o-assunto-247-os-erros-do-mexico-3o-em-mortes-por-covid.ghtml}{Notícia - O Assunto \#247: Os erros do México, 3º em mortes por Covid}
    \item \href{https://news.google.com/covid19/map?hl=pt-PT&mid=\%2Fm\%2F0b90_r&gl=PT&ceid=PT\%3Apt-150}{Dados - Google notícias}
    \item \href{https://ourworldindata.org/coronavirus}{Dados - Our World in Data}
    \item \href{https://iopscience.iop.org/article/10.1088/1478-3975/abb115}{Artigo principal 1 - Using posterior predictive distributions to analyse epidemic models: COVID-19 in Mexico City}
    \item \href{https://www.sciencedirect.com/science/article/pii/S0025556420300596}{Artigo principal 2 - Modeling behavioral change and COVID-19 containment in Mexico: A trade-off between lockdown and compliance}
    \item \href{https://www.sciencedirect.com/science/article/pii/S0960077920303453}{Artigo principal 3 - Modeling and prediction of COVID-19 in Mexico applying mathematical and computational models}
\end{enumerate}

\end{document}
